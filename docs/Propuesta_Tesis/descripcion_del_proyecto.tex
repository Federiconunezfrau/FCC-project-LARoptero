\section{Descripción del Proyecto}

%El presente trabajo se enmarca en un proyecto más amplio, dirigido por el Laboratorio de Automática y Robótica de la FIUBA (LAR), el cual consiste en el desarrollo de un vehículo hexarotor tolerante a fallas. Esto contempla el diseño mecánico del vehículo, el diseño de la computadora de vuelo y su correspondiente software, el diseño e implementación de los algoritmos de control adecuados, el diseño de un sistema de estimación y control de velocidad, el diseño de algoritmos de evasión de obstáculos y el diseño de técnicas de detección e identificación de fallas en pleno vuelo.\\

%En el marco de este proyecto, el presente trabajo de Tesis tiene por objetivo el diseño y construcción de la computadora de vuelo de bajo nivel. Esto contempla el diseño de la plataforma de hardware PCB según los requerimientos del proyecto en el que se enmarca, y el diseño e implementación de la técnica utilizada para la tolerancia a fallas a partir del uso de redundancias de hardware.

%En los últimos años se ha incrementado mucho la presencia de vehículos aéreos no tripulados en espacio aéreo civil. 


En los últimos años ha habido un incremento en el uso de vehículos aéreos no tripulados para aplicaciones fuera del ámbito militar, como por ejemplo búsqueda y rescate, uso en construcciones e inspección, agricultura de precisión, vigilancia, entre otros \cite{8682048}. Teniendo en cuenta la importancia que han tomado en distintas actividades, además del hecho de que en muchas de estas aplicaciones estos sobrevuelan zonas donde circulan personas, resulta mandatorio garantizar cierto grado de confiabilidad en su funcionamiento. En vehículos aéreos tripulados como aviones comerciales y militares, es común que para ello se utilicen técnicas de tolerancia a fallas a partir de redundancias. Esto mismo ocurre con vehículos aéreos no tripulados de uso militar, aunque no es tan común en aquellos de uso civil y comercial.\\

En este contexto, el objetivo de este trabajo de Tesis es el diseño y construcción de una computadora de vuelo, con la capacidad de implementar mecanismos de tolerancia a fallas, a partir de redundancias, utilizando componentes \textit{commercial off-the-shelf} (COTS) \cite[p.~26]{EASAcots}.