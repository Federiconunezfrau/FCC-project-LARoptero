\section{Objetivo General y Objetivos Particualres}

El presente trabajo de Tesis tiene por objetivo el diseño y construcción de una computadora de vuelo de bajo nivel, a ser utilizada en un vehíuclo aéreo hexarotor, no tripulado. Como aspecto particular, esta debe contar con la capacidad de tolerar ciertas fallas de hardware que puedan ocurrir en pleno vuelo. Lo que se busca, es que estas fallas no impacten en la misión del vehículo y que puedan ser detectadas lo antes posible para tomar una acción.\\

El presente trabajo se enmarca en un proyecto más amplio, dirigido por el Laboratorio de Automática y Robótica de la FIUBA (LAR), el cual consiste en el desarrollo de un vehículo hexarotor tolerante a fallas. Esto contempla el diseño mecánico del vehículo, el diseño de la computadora de vuelo y su correspondiente software, el diseño e implementación de los algoritmos de control adecuados, el diseño de un sistema de estimación y control de velocidad, el diseño de algoritmos de evasión de obstáculos y el diseño de técnicas de detección e identificación de fallas en pleno vuelo.

Los objetivos particulares son:

\begin{itemize}
    \item Entender el estado del arte, en cuanto a tolerancia de fallas en vehículos aéreos no tripulados.
    \item Diseño y construcción de la computadora de vuelo.
    \item Pruebas del correcto funcionamiento de la misma.
    \item Desarrollo de un prototipo que ejecute un mecanismo de tolerancia a fallas.
    \item Validación del prototipo, simulando distintos tipos de fallas en el vehículo.
\end{itemize}