\section{Descripción del Alcance del Proyecto y Planteo de los Mecanismos que se Utilizarán para Verificar la Calidad de los Resultados Obtenidos}

La principal tarea consiste en el diseño y el desarrollo de la computadora de vuelo, teniendo en cuenta los requerimientos planteados por el proyecto en el que se enmarca. Para ello se realizarán las siguientes tareas:

\begin{itemize}
    \item Recolectar información sobre las normas comerciales para el hardware, respecto a requerimientos de cada uno de los posibles integrados, componentes, y diseño de un PCB. %Observar requerimientos para conectores, redundancia, normas de sellado, etc.
    \item Selección de componentes físicos para una computadora de vuelo. Esto comprende los distintos tipos de sensores necesarios, los módulos que debe controlar la computadora de vuelo y las interfaces de comunicación externas. Particularmente, se tendrá en cuenta la necesidad del stacking de múltiples controladoras de vuelo para redundancia.
    \item Diseño del circuito esquemático y desarrollo de la placa de circuito impreso de la computadora de vuelo, teniendo en cuenta aspectos de manufacturabilidad.
    \item Análisis de los requerimientos y las características de un sistema tolerante a fallas de tiempo real.
    \item Construcción de un prototipo que demuestre la capacidad de tolerar fallas, en la computadora de vuelo. 
\end{itemize}
