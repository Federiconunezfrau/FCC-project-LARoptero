\section{Definición de la Necesidad y Evaluación Preliminar de las Soluciones Existentes}

%Los sistemas tolerantes a fallas aceptan que estas puedan ocurrir, pero implementan mecanismos que permiten continuar con su normal funcionamiento. La principal técnica de tolerancia a fallas es el uso de redundancias \cite{lala1994architectural}. Esto quiere decir que se replica el hardware en el sistema y cada réplica realiza la misma tarea en paralelo. Típicamente, existe una comparación entre los resultados calculados por cada una de las réplicas, con el objetivo de detectar y enmascarar las fallas. El hecho de duplicar o hasta triplicar el hardware trae consigo un incremento en el costo del vehículo.\\

%{\color{red} La verdadera necesidad es que el LAR quiere hacer el proyecto y que por ende necesita la computadora de vuelo tolerante a fallas.}

El LAR se encuentra trabajando en un proyecto de diseño y construcción de un vehículo aéreo no tripulado con capacidad de tolerancia a fallas. Este proyecto resulta muy amplio, ya que abarca distintos aspectos del vehículo. Uno de ellos comprende la necesidad de un módulo dedicado al control de sus motores y estimación de la pose a partir de mediciones de sensores, denominado comptuadora de vuelo. Como aspecto a destacar, esta debe integrar mecanismos que permitan implementar distintos tipos de técnicas de tolerancia a fallas. Por ejemplo, una falla en un sensor o en un cálculo de la ley de control, podría traer consecuencias como el fracaso de la misión. Teniendo en cuenta que estos vehículos cada vez más sobrevuelan espacio aéreo civil, resulta necesario garantizar cierto grado de confiabilidad.\\

Se buscaron trabajos similares que preferentemente utilicen componentes COTS. Dos de los trabajos que se tendrán en cuenta para el desarrollo de esta Tesis son \cite{hiergeist2018implementation} y \cite{zhang2020architecture}. Ambos implementan mecanismos de tolerancia a fallas a través del intercambio de información entre los nodos redundantes, como puede ser una medición de un sensor o el resultado del cálculo de la ley de control. En el caso en el que alguno de los nodos presente algún comportamiento anómalo, este entregará información que no se corresponde con la entregada por los demás nodos. Esto es algo típico de sistemas redundantes, como es el caso de los aviones comerciales y militares, donde las fallas son detectadas utilizando algoritmos de votación de la información entregada por cada nodo \cite[p.~217]{collinson2023introduction}. En \cite{hiergeist2018implementation} se implementa una arquitectura con redundancia cuádruple para un UAV. Estos típicamente realizan las tareas de relevamiento de datos de sensores, cálculo de la ley de control y aplicación de resultados a sus actuadores, de manera cíclica. La tolerancia a fallas se implementa a través del intercambio de información entre los 4 nodos, buscando lograr un consenso. Esto quiere decir que todos los nodos deben coincidir por ejemplo, en el valor de la medición de un sensor o en el resultado de un cálculo de la ley de control. En \cite{zhang2020architecture}, el trabajo también implementa la tolerancia a fallas a través del intercambio de sus nodos. Este tiene la particularidad de hacer énfasis en una arquitectura gobernada por el tiempo, es decir que el vehículo funciona de una forma predefinida, lo que le da mayor determinismo y seguridad al sistema. Esto se traduce en que los nodos trabajan de manera sincronizada.\\

El desarrollo del presente trabajo tendrá en cuenta los requerimientos típicos de un sistema tolerante a fallas, de tiempo real. A su vez se considerarán los requerimientos planteados por el LAR, correspondientes al proyecto en el que este trabajo se enmarca. Esto comprende los sensores e interfaces de comunicación necesarias de propósito general, además de la interfaz de comunicación utilizada para implementar el mecanismo de tolerancia a fallas.