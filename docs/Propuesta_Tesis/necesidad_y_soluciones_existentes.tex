\section{Definición de la Necesidad y Evaluación Preliminar de las Soluciones Existentes}



%Los sistemas tolerantes a fallas aceptan que estas puedan ocurrir, pero implementan mecanismos que permiten continuar con su normal funcionamiento. La principal técnica de tolerancia a fallas es el uso de redundancias \cite{lala1994architectural}. Esto quiere decir que se replica el hardware en el sistema y cada réplica realiza la misma tarea en paralelo. Típicamente, existe una comparación entre los resultados calculados por cada una de las réplicas, con el objetivo de detectar y enmascarar las fallas. El hecho de duplicar o hasta triplicar el hardware trae consigo un incremento en el costo del vehículo.\\

Se buscaron trabajos similares que preferentemente utilicen componentes COTS. Dos de los trabajos que se tendrán en cuenta para el desarrollo de esta Tesis son \cite{hiergeist2018implementation} y \cite{zhang2020architecture}. Ambos implementan mecanismos de tolerancia a fallas, utilizando componentes COTS, a través del intercambio de información entre los nodos redundantes, buscando llegar a un consenso respecto a algún dato, como puede ser una medición de un sensor. {\color{red} agregar que esto es común en aviones. Citar algo}. En \cite{hiergeist2018implementation} se implementa una arquitectura con redundancia cuádruple para un UAV. Estos típicamente realizan las tareas de relevamiento de datos de sensores, cálculo de la ley de control y aplicación de resultados a sus actuadores, de manera periódica. La tolerancia a fallas se implementa a través del intercambio de información entre los 4 nodos, buscando lograr un consenso. Esto quiere decir que todos los nodos deben coincidir por ejemplo, en el valor de la medición de un sensor o en el resultado de un cálculo de la ley de control. Por otro lado, en \cite{zhang2020architecture}, el trabajo también implementa la tolerancia a fallas a través del intercambio de sus nodos. Este tiene la particularidad de hacer énfasis en una arquitectura gobernada por el tiempo, es decir que el vehículo funciona de una forma predefinida, lo que le da mayor determinismo y seguridad al sistema. Esto se traduce en que los nodos trabajan de manera sincronizada.\\

El desarrollo del presente trabajo tendrá en cuenta los requerimientos típicos de un sistema tolerante a fallas, de tiempo real. Se tendrán en cuenta los requerimientos del laboratorio en cuanto a los sensores e interfaces de comunicación necesarias, en particular en la interfaz de comunicación necesaria para implementar el mecanismo de tolerancia a fallas.