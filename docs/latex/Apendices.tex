\appendix
\appendixpage
\addappheadtotoc

\section*{Apéndice A: Circuito Esquemático}
\addcontentsline{toc}{section}{Apéndice A: Circuito Esquemático}\label{appendix:circuito_esquematico}

\begin{center}
    \includesvg[scale=0.65, inkscapelatex=false]{esquematico_1}
\end{center}

\begin{center}
    \includesvg[scale=0.65, inkscapelatex=false]{esquematico_2}
\end{center}

\begin{center}
    \includesvg[scale=0.65, inkscapelatex=false]{esquematico_3}
\end{center}

\begin{center}
    \includesvg[scale=0.65, inkscapelatex=false]{esquematico_4}
\end{center}

\begin{center}
    \includesvg[scale=0.65, inkscapelatex=false]{esquematico_5}
\end{center}

\begin{center}
    \includesvg[scale=0.65, inkscapelatex=false]{esquematico_6}
\end{center}

%\textbf{{\color{red} COMPLETAR}}

\section*{Apéndice B: PCB Final}
\addcontentsline{toc}{section}{Apéndice B: PCB Final}\label{appendix:PCB}

Se adjunta el PCB final. Todos los componentes se encuentran en la misma cara. Para apreciar el trazado de pistas y los cambios de capa, se adjunta una imagen donde se han ocultado los planos de GND y 3V3.

\begin{center}
    \includesvg[scale=1.4, inkscapelatex=false]{PCB_2}
\end{center}

%\textbf{{\color{red} COMPLETAR}}