\section{Introducción}\label{sec:Introduccion}

%Introducción: acá se pone en contexto, por ejemplo se puede describir el proyecto del LAR y llegar a que uno de los elementos del drone es la computadora de vuelo, pertinente a este trabajo. También acá, tendría que mencionar el por qué de la tolerancia a fallas. Podría dar algún ejemplo de algún accidente que haya ocurrido. Por último, ejemplos de usos y aplicaciones civiles. Acá se justifica el por qué de la tolerancia a fallas. En las siguientes secciones se explica cómo se aborda y qué se desarrolló.

Los vehículos aéreos no tripulados (UAVs) originalmente fueron desarrollados para uso en aplicaciones militares. Al no contar con un piloto ni con una tripulación a bordo, estos permiten llevar adelante tareas peligrosas, que pueden llegar a poner en riesgo la vida de las personas. Sumado a esto, el desarrollo y mantenimiento de este tipo de vehículos resulta menos costoso frente al de un avión tripulado \cite[p.~490]{collinson2023introduction}. Estos factores fueron claves como motivación para el desarrollo de este tipo de vehículos.

Algunos de los principales usos en el ámbito militar son reconocimiento, vigilancia y monitoreo. Al contar con distintos tipos de sensores, como cámaras, sensores infrarrojos, entre otros, estos pueden recolectar información en zonas hostiles, de manera económica y segura.

A partir de los avances en la tecnología y la reducción en los costos de fabricación, tanto para sensores como componentes en general, este tipo de vehículos han comenzado a ser utilizados para fines civiles y comerciales. De esta manera, las mismas ventajas por las cuales comenzaron a ser utilizados en el ámbito militar, despertaron el interés de distintos sectores por fuera de este. Hoy en día estos vehículos son utilizados para distintas aplicaciones civiles y comerciales. Algunas de ellas son:

\begin{itemize}
    \item Búsqueda y Rescate: En escenarios de desastres naturales, los UAVs son utilizados para tomar imágenes y videos de las zonas afectadas o búsqueda de personas. El uso de un UAV frente a un avión o helicóptero, elimina potenciales riesgos para el piloto y la tripulación, además de reducir el costo de la operación. Un ejemplo relevante de este uso de UAVs fue durante el accidente nuclear de Fukushima del año 2011 ocurrido en Japón. Debido a la radiación de la zona, el uso de UAVs fue necesario para la recolección de imágenes y video. Además, se utilizó un UAV de la Fuerza Aérea de Estados Unidos equipado con un sensor infrarrojo para conocer la temperatura de los reactores nucleares \cite{adams2011survey}.
    \item Teledetección: El bajo coste de los UAVs permite obtener gran cantidad de datos para distintas investigaciones del suelo y del medio ambiente. En \cite{villa2016overview} se puede encontrar un trabajo en donde se utilizaron vehículos aéreos no tripulados para realizar un muestreo ambiental en el Ártico, además de estudios acerca de la temperatura de la superficie del océano. En \cite{mcgonigle2008unmanned} se presenta un trabajo donde se utiliza un UAV para realizar mediciones sobre gases volcánicos. Utilizando distintos tipos de espectrómetros y sensores electroquímicos, se analizan las concentraciones de dióxido de carbono y dióxido de azufre.
    \item Inspección en Infraestructura y Construcciones: Utilizando UAVs es posible realizar tareas de inspección para encontrar problemas en la red de distribución de energía eléctrica \cite{luque2014power}. Existen algunas empresas que se dedican a estas actividades de inspección, como por ejemplo Cyberhawk. Esta además ofrece otros servicios destinados a distintos sectores industriales, como monitoreo en construcciones y de generadores eólicos.
    \item Agricultura de Precisión: El uso de UAVs en este sector está destinado a mejorar el rendimiento del cultivo, a través de distintas actividades como el riego programado, la detección temprana de pestes y el sensado de la textura del suelo. Este último puede usarse como indicador de la calidad del suelo para cultivo.
    \item Vigilancia: En algunos países los UAVs son utilizados para patrullar y controlar las fronteras, por ejemplo para detectar situaciones de tráfico ilegal.
\end{itemize}

El factor común a todas estas aplicaciones es que la incorporación de los UAVs en el ámbito civil y comercial ha abierto oportunidades para realizar tareas y actividades que de otra forma serían muy costosas y/o riesgosas para las personas.

Teniendo en cuenta la importancia que han tomado, además del hecho de que en muchas de estas aplicaciones estos sobrevuelan zonas donde circulan personas, resulta mandatorio garantizar cierto grado de confiabilidad en su funcionamiento. Este es un aspecto que caracteriza la capacidad de un sistema para funcionar correctamente durante un período de tiempo, donde ``correctamente'' se refiere a cumplir con la tarea para el cual fue diseñado. Para el caso de un UAV, el hecho de volar en espacio aéreo civil puede llegar a causar daño físico a personas, si es que un vehículo presenta una falla y por ejemplo pierde el control. Otra de las posibles consecuencias tiene que ver con los costos que puede ocasionar una falla en una misión relacionada a una actividad laboral. El hecho de tener que repetir la misión puede traer mayores costos para la actividad en cuestión.

Un UAV típicamente se compone de distintos elementos. Cada uno de ellos es susceptible de manifestar distintos tipos de fallas que pueden afectar al vehículo de distintas maneras. Algunos de los elementos más importantes son la estructura mecánica, el sistema de batería y alimentación, los motores y actuadores, un sistema de comunicación remota con un piloto y la computadora de vuelo. En este trabajo se abordarán aspectos relacionados a fallas relacionadas con este último.

\subsection{Computadora de Vuelo}

La computadora de vuelo se compone de una unidad de procesamiento que realiza la adquisición de datos de sensores y ejecuta los algoritmos necesarios para estabilizar y controlar el vehículo. Suelen incorporar una variedad de sensores a bordo, siendo el más común de ellos la Unidad de Medición Inercial (IMU), compuesta por acelerómetros y giróscopos triaxiales. A su vez, suelen disponer de magnetómetros triaxiales, barómetros, GPS, LiDARs, sensores ultrasónicos, sensores de flujo óptico, entre otros diferentes tipos de sensores de velocidad y distancia. Sumado a esto, suelen contar con distintas interfaces para comunicación con otros módulos externos, como pueden ser sensores, actuadores u otros que sean de uso específico para cumplir con la misión del vehículo.

Los datos de los sensores son utilizados para ejecutar los distintos algoritmos de navegación y control. Periódicamente se adquieren mediciones de los distintos sensores del vehículo, se procesan dichos datos para luego aplicar acciones de control a los distintos actuadores, es decir actuar sobre los motores. De esta manera se logra que el vehículo recorra una trayectoria previamente configurada, o bien que responda adecuadamente a los comandos enviados por un piloto a distancia.

%Se encarga de adquirir datos de los sensores, ejecutar los algoritmos necesarios para estabilizar y controlar el vehículo y aplicar los resultados a sus actuadores, es decir sobre los motores del vehículo.


%Para ello, periódicamente se adquieren mediciones de los distintos sensores del vehículo, se procesan dichos datos para luego aplicar acciones de control a los distintos actuadores, es decir actuar sobre los motores. De esta manera se logra que el vehículo recorra una trayectoria previamente configurada, o bien que responda adecuadamente a los comandos enviados por un piloto a distancia.


%La computadora de vuelo es el elemento central en un vehículo aéreo no tripulado. Esta centraliza toda la información. Se encarga de realizar la adquisición de datos de sensores y ejecutar los algoritmos necesarios para estabilizar y controlar el vehículo. Para ello, periódicamente se adquieren mediciones de los distintos sensores del vehículo, se procesan dichos datos para luego aplicar acciones de control a los distintos actuadores, es decir actuar sobre los motores. De esta manera se logra que el vehículo recorra una trayectoria previamente configurada, o bien que responda adecuadamente a los comandos enviados por un piloto a distancia.

%\subsection{Características de la Computadora de Vuelo a Desarrollar}
%\subsection{Características y Objetivos Particulares}
\subsection{Tolerancia a Fallas y Redundancias}

Resulta evidente que la computadora de vuelo es el elemento central en un vehículo aéreo no tripulado, por lo que una falla puede traer consecuencias graves. Por ejemplo, una falla en una lectura de un sensor, puede resultar en una mala estimación de la posición del vehículo o incluso decantar en la pérdida de control de este. En vehículos aéreos tripulados como aviones comerciales y militares, es común que se utilicen distintas técnicas de tolerancia a fallas para mejorar la confiabilidad. Esto mismo ocurre con vehículos aéreos no tripulados de uso militar, aunque no es tan común en aquellos de uso civil y comercial. 

La computadora de vuelo de este trabajo se desarrollará teniendo como objetivo la necesidad de implementar técnicas de tolerancia a fallas a partir del uso de redundancias. %Esta es la principal técnica utilizada en sistemas de tiempo real \cite{nelson1990fault}
Esto implica que las tareas realizadas se ejecuten utilizando réplicas de hardware. En el eventual caso de que una de estas réplicas presente una falla y el sistema la detecte, se deberá tomar una acción, como por ejemplo descartar la réplica fallada y utilizar alguna de las demás réplicas sin fallas. 

El solo hecho de incoporar redundancias en un sistema no equivale a incrementar la confiabilidad. Es necesario incorporar mecanismos que administren correctamente los aspectos relacionados al manejo de las redundancias. La forma más común de detectar fallas es realizando comparaciones entre los resultados calculados por cada réplica. Por ejemplo, si se contara con un sistema con doble redundancia, podría concluirse que alguna de las dos réplicas presentó una falla a partir de la comparación de los resultados obtenidos por cada una de ellas. Sin embargo, a priori no podría decirse cuál de estas fue la que falló. Ambas réplicas deberían ejecutar una rutina auxiliar que realice una verificación interna. Teniendo en cuenta que el sistema de control de vuelo del UAV es un sistema de tiempo real, la ejecución de esta rutina podría perjudicar su determinismo temporal y poner en riesgo la estabilidad del vehículo. Este ejemplo ilustra la necesidad de analizar y administrar correctamente los aspectos relacionados al manejo de las redundancias.

En este trabajo se diseñará el circuito correspondiente a la computadora de vuelo y se fabricarán tres réplicas, cada una en su propio PCB. Luego de verificar el correcto funcionamiento de cada una de estas por separado, se procederá a proponer una arquitectura distribuida, para administrar las redundancias. Finalmente, se utilizarán las tres placas para demostrar el funcionamiento del sistema redundante.

%A partir de la comparación de los resultados obtenidos por cada réplica, es posible detectar fallas y tomar acciones que permitan continuar con la misión del vehículo.

%Se toman las consideraciones necesarias para la conexión y gestión de la información entre las tres réplicas. Se plantea una arquitectura distribuida, donde cada nodo recibe los datos adquiridos por las demás y ejecuta un algoritmo para corroborar la consistencia de los mismos. Para validar el funcionamiento, se realiza una prueba experimental donde se detecta la falla de una unidad inercial.

%En este contexto, el objetivo de este trabajo de Tesis es el diseño y construcción de una computadora de vuelo, con la capacidad de implementar mecanismos de tolerancia a fallas, a partir de redundancias, utilizando componentes commercial off-the-shelf (COTS).

%En este contexto, el objetivo de este trabajo de Tesis es el diseño y construcción de una computadora de vuelo, con la capacidad de implementar mecanismos de tolerancia a fallas, a partir de redundancias, utilizando componentes commercial off-the-shelf (COTS).

%Un UAV típicamente se compone de varios elementos, como una estructura mecánica, un sistema de batería y alimentación para el vehículo, sus motores y actuadores, un sistema de comunicación remota con un piloto y una computadora de vuelo. Si bien cada uno de estos es susceptible de manifestar distintos tipos de fallas que puedan afectar al vehículo de distintas formas, en este trabajo se abordarán aspectos relacionados a fallas en la computadora de vuelo. Este es el elemento central...

%En los últimos años se ha incrementado mucho la presencia de UAVs en espacio aéreo civil. Debido a esto, se plantea que los UAVs deberían presentar características que permitan un funcionamiento correcto, tolerante a fallas. Como consecuencias posibles, el hecho de volar en espacio aéreo civil puede llegar a causar daño físico a personas, si es que un vehículo presenta una falla y por ejemplo pierde el control. Otra de las posibles consecuencias tiene que ver con los costos que puede ocasionar una falla en una misión relacionada a una actividad laboral. El hecho de tener que repetir la misión puede traer mayores costos para la actividad en cuestión. El objetivo del diseño tolerante a fallas consiste en mejorar la confianza (Dependability) del sistema, apuntando a que este pueda seguir ejecutando su función de manera correcta a pesar de la presencia de una cierta cantidad de fallas [54]. De esta última expresión se puede tomar una definición de lo que es un sistema tolerante a fallas

%En vehículos aéreos tripulados como aviones comerciales y militares, es común que para ello se utilicen técnicas de tolerancia a fallas a partir de redundancias. Esto mismo ocurre con vehículos aéreos no tripulados de uso militar, aunque no es tan común en aquellos de uso civil y comercial. En este contexto, el objetivo de este trabajo de Tesis es el diseño y construcción de una computadora de vuelo, con la capacidad de implementar mecanismos de tolerancia a fallas, a partir de redundancias, utilizando componentes commercial off-the-shelf (COTS).

%En los últimos años ha habido un incremento en el uso de vehículos aéreos no tripulados... Esto implica que se requiere cierto grado de confiabilidad...\\

%La computadora de vuelo se compone de ...\\

%Un sistema crítico (en inglés \textit{safety-critical system}) es aquel en el que un fracaso puede traer consigo consecuencias catastróficas, como daños graves a personas, pérdida de vidas o daños importantes al medio ambiente \cite[p.~271]{kopetz-2011}. Por su definición, un sistema puede caracterízarse como crítico o no crítico antes de su implementación, ya que lo que determina esto es simplemente el uso que se le dará. Algunas áreas típicas donde se pueden encontrar estos sistemas son aviónica, equipamiento médico y energía nuclear.\\

%\textbf{{\color{red} COMPLETAR}}