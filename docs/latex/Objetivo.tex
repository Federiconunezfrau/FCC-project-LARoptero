\section{Objetivo}

%Objetivo: a dónde querés llegar con este trabajo. Acá describo en 1 ó 2 páginas cuál es el objetivo, es decir: 
%- El análisis de las características de sistemas críticos en general. 
%- La construcción de la placa para usar como computadora de vuelo.
%- El desarrollo del prototipo, simulando las fallas.

El presente trabajo de Tesis tiene por objetivo el diseño y construcción de una computadora de vuelo de bajo nivel, a ser utilizada en un vehículo aéreo hexarotor, no tripulado. Como aspecto particular, esta debe contar con la capacidad de tolerar ciertas fallas de hardware que puedan ocurrir en pleno vuelo. Lo que se busca, es que estas fallas no impacten en la misión del vehículo y que puedan ser detectadas lo antes posible para tomar una acción.

En primera medida, se hace un análisis e investigación acerca del estado del arte, para vehículos aéreos no tripulados de caracter comercial, principalmente drones. El objetivo es conocer los mecanismos de seguridad que se implementan en este tipo de vehículos, tanto de hardware como de software. Por otro lado, se busca conocer cuáles son las normas actuales, pertinentes al uso y comercialización de vehículos aéres no tripulados, principalmente drones.

Lo siguiente es el desarrollo de una computadora de vuelo. Esto comprende la definición de los requerimientos de la misma, principalmente de hardware en cuanto a sensores, conectores y funcionalidades deseadas. A partir de estos, se hace una investigación de la variedad de componentes disponibles. Luego, se pasa a una etapa de selección de los componentes a utilizar. Por último, se define un circuito esquemático y se diseña un PCB, el cual será enviado a fabricación.

Para abordar la tolerancia a fallas de hardware de la computadora de vuelo, se plantea utilizar técnicas que involucran la redundancia, tanto de hardware como de software. Para ello, se lleva a cabo una investigación de las técnicas comúnmente utilizadas en el sector aeronáutico para tolerancia de fallas. Finalmente, se define un esquema y una arquitectura a utilizar como mecanismo de tolerancia a fallas.

Para demostrar los resultados, se presentan resultados de pruebas de control de un motor en una arquitectura redundante, sobre la cual se simula la manifestación de distintos tipos de fallas. Se presentan los resultados y la respuesta del sistema.