\section{Conclusiones}

% Tolerancia a Fallas
En la bibliografía se destaca que la tolerancia a fallas a partir de redundancias es una técnica amplialmente utilizada. Esta se presta a ser utilizada en sistemas de tiempo real ya que no requiere una gran carga computacional y con una simple comparación de resultados pueden detectarse fallas. Para que la tarea de tolerancia a fallas sea efectiva, es necesario que el sistema distribuido trabaje de manera sincronizada. Sumado a esto, es necesario utilizar un bus de comunicaciones como medio para el intercambio de información entre réplicas, donde cada una de ellas tenga predefinido en qué momentos pueden enviar mensajes y en cuáles no. Estos últimos 2 aspectos garantizan que todas las réplicas tienen la misma información acerca del sistema.

% Computadora de Vuelo
En cuanto al desarrollo de la computadora de vuelo, pudieron fabricarse en total 3 placas, para las cuales pudo verificarse su correcto funcionamiento de forma individual. La gran cantidad de conectores presentes en la placa, suamdo al requerimiento de posicionamiento del sensor IMU y las limitaciones de fabricación que obligaron a colocar todos los componentes de un solo lado, presentaron un desafío a la hora de diseñar el PCB. El hecho de haber elegido un PCB de 4 capas fue una decisión que permitió que esto pueda cumplirse, cumpliendo al mismo tiempo con la necesidad de mantener unas dimensiones reducidas. A pesar de no haber podido actualizar el microcontrolador de la computadora de vuelo con una de las alternativas evaluadas, pudo obtenerse un diseño funcional, con sensores de mejor rendimiento y con las interfaces y funcionalidades requeridas para ser utilizada en vehículos autónomos.

% Firmware y Pruebas
A partir de un firmware desarrollado, se ha demostrado la capacidad de la computadora de vuelo de ser utilizada con una arquitectura sincronizada de 3 réplicas en paralelo. Utilizando un método de sincronización muy simple pudo lograrse una precisión de las decenas de $\mu$s, aunque en la bibliografía presentada en este trabajo se pueden encontrar algoritmos que permiten mejorar esto. Además, este debería ser distribuido, evitando así que ninguna de las réplicas se vuelva un punto singular de fallas. A pesar de esto, la implementación utilizada permitió demostrar las capacidades de la placa para funcionar con la arquitectura necesaria para comparar resultados de navegación continuamente y detectar fallas durante el funcionamiento. 

% Trabajo futuro
Los resultados presentados en este trabajo demuestran las capacidades de la computadora de vuelo para ser utilizada con la arquitectura propuesta para la tolerancia a fallas. Sin embargo, aún quedan muchos aspectos por resolver para que el sistema propuesto pueda ser utilizado en el vuelo de un vehículo. Aquí solo se ha utilizado el sensor IMU, pero deberían incorporarse otros sensores como el barómetro, el magnetómetro y otros adicionales externos a la placa. Además, todas las mediciones que estos entregan serán utilizadas por el sistema INS para realizar una estimación más completa de la pose del vehículo, algo que requiere una gran carga computacional y que debe evaluarse cómo puede ser integrado en un esquema estático como el que se propone en este trabajo. Posiblemente, esto requiera que deban realizarse modificaciones a las implementaciones ya existentes, no en cuanto a los cálculos realizados, pero sí en la forma en que estos se ejecutan, buscando obtener un algoritmo con un tiempo de ejecución balanceado. 

Además del uso de un algoritmo de sincronización distribuido, restarían implementarse otras funcionalidades que favorezcan la seguridad. Por un lado, sería necesario el agregado de un sistema capaz de monitorear que cada réplica respete los tiempos de su scheduler, de manera que si una de ellas no lo cumple, pueda tomarse una acción correctiva. Esto sería necesario, ya que la péridda de sincronización de una placa puede además afectar el funcionamiento de todo el sistema, si es que perjudica las comunicaciones presentes en el bus. Por otro lado, este aspecto también podría mejorarse si se utiliza el 2do bus CAN presente en la placa.