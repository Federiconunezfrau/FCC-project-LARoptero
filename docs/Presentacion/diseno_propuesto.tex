\subsection{Dimensiones de la Antena}
%% ========================================================
\begin{frame}{Ancho $W$}
    \begin{itemize}
        \item<1-> Se utilizó la expresión del libro de Balanis.
        \item<2-> Este dimensionamiento da un buen rendimiento de la antena.
    \end{itemize}

    \begin{center}
        \onslide<3->$W = \frac{c_0}{2 \ f_{010}} \sqrt{\frac{2}{\varepsilon_r+1}}$
    \end{center}

    \begin{itemize}
        \item<4-> Utilizando $f_{010} = 1575.42 \ MHz$ y $\varepsilon_r = 4.4 \Rightarrow W = 57.92 \ mm$
    \end{itemize}
\end{frame}

%% ========================================================
\begin{frame}{Longitud $L$}

    \begin{itemize}
        \item<1-> La longitud se despeja a partir de la ecuación de la frecuencia de resonancia del parche: $f_{010} = \frac{1}{2 \sqrt{\mu \epsilon}} \frac{1}{L}$
        \item<2-> Modelo de Cavidad, se asume que las líneas de campo son verticales dentro del parche.
        \item<3-> Fenómeno de \textit{fringing}:
    \end{itemize}
    \begin{columns}
        \column{0.5\textwidth}
            \begin{itemize}
                \item<4-> $f_{010} = \frac{1}{2 \sqrt{\mu \epsilon}} \frac{1}{L + 2 \Delta L}$
                \item<5-> $L = \frac{c_0}{2 \ f_{010} \sqrt{\varepsilon_{r,eff}}} - 2 \Delta L$
                \item<6-> Se utiliza la expresión del libro de Balanis para $\Delta L$.
                \item<7-> $L = 44.92 \ mm$
            \end{itemize}
        \column{0.5\textwidth}
            \onslide<3->\includegraphics[width=\textwidth]{img/fringing.png}
    \end{columns}
\end{frame}

%% ========================================================
\begin{frame}{Dimensiones del Plano de tierra}
    \begin{itemize}
        \item Se toma como extensión del plano de tierra: $\lambda/4$.
    \end{itemize}

    \begin{center}
        \includegraphics[width=0.5\textwidth]{img/extensiones_plano_tierra.png}
    \end{center}
\end{frame}

%% ========================================================
\begin{frame}{Inserción del Microstrip en el Parche}

\begin{columns}
    \column{0.5\textwidth}
        \begin{itemize}
            \item<1-> $y_0 = arccos\left( \sqrt{\frac{50 \Omega}{R_{in}(y_0 = 0)}} \right) \frac{L}{\pi}$
            \item<2-> $R_{in}(y_0 = 0)$ se calcula aproximadamente. 
            \item<3-> El valor de $y_0$ se modificará hasta lograr la adaptación justa.
        \end{itemize}
    \column{0.5\textwidth}
        \includegraphics[width=\textwidth]{img/parche_hendidura.png}
\end{columns}
\end{frame}

%% ========================================================
\subsection{Simulaciones}
\begin{frame}{Primera Simulación en FEKO}
    \begin{itemize}
        \item<1-> Simulación con las dimensiones calculadas.
        \item<2-> Se hace la simulación del parámetro $s_{11}$ en FEKO.
    \end{itemize}
    \begin{columns}
        \column{0.4\textwidth}
            \begin{itemize}
                \item<3-> $W = 57.92 \ mm$
                \item<3-> $L = 44.92 \ mm$
                \item<3-> $y_0 = 16 \ mm$
            \end{itemize}
        \column{0.6\textwidth}
            \includegraphics[width=\textwidth]{img/FEKO_dimensiones_teoricas.png}
    \end{columns}
\end{frame}

%% ========================================================
\begin{frame}{Primera Simulación: ROE}

\begin{itemize}
    \item Se exportan los datos del parámetro $s_{11}$, para graficar la ROE:
\end{itemize}

\begin{center}
    \includegraphics[width=\textwidth]{img/ROE_dimensiones_teoricas.png}
\end{center}

\begin{itemize}
    \item<2-> Corregir resonancia: $6.3 \ MHz$ $\Rightarrow$ modificar $L$.
    \item<3-> Mejorar la adaptación, ROE muy grande $\Rightarrow$ modificar $y_0$.
\end{itemize}
    
\end{frame}

%% ========================================================
\begin{frame}{Primera Simulación: ROE}
    \begin{itemize}[<+->]
        \item Para corregir las dimensiones, se hicieron modificaciones manuales.
        \item Se observó que las dimensiones calculadas de forma teórica requieren modificaciones de unos pocos milímetros.
        \item Se utilizó la herramienta de optimización de FEKO.
    \end{itemize}
\end{frame}

%% ========================================================
\begin{frame}{Primera Simulación: ROE}
    \begin{itemize}
        \item<1-> Modificar $y_0$ también modifica la resonancia:
    \end{itemize}
    \vspace{1cm}
    \begin{columns}
        \column{0.5\textwidth}
            \onslide<3->\includegraphics[width=\textwidth]{img/ROE_vs_L.png}
        \column{0.5\textwidth}
            \onslide<2->\includegraphics[width=\textwidth]{img/ROE_vs_y0.png}
    \end{columns}
    
\end{frame}

%% ========================================================
\begin{frame}{Primera Simulación: ROE}
    \begin{columns}
        \column{0.4\textwidth}
             \begin{itemize}
                \item Optimización doble, utilizando FEKO.
                \item Se ajustan ambos parámetros al mismo tiempo.
                \item Lograr la mejor adaptación, a la frecuencia deseada.
                \item<2-> $L = 45.16 \ mm$
                \item<2-> $y_0 = 14.23 \ mm$
            \end{itemize}
        \column{0.6\textwidth}
            \includegraphics[width=\textwidth]{img/resultados_optimizacion_doble.png}
    \end{columns}
\end{frame}

%% ========================================================
\begin{frame}{Primera Simulación: ROE}
\begin{itemize}
    \item<1-> Se repite la simulación con las modificaciones:
\end{itemize}
    \begin{center}
        \onslide<2->\includegraphics[width=\textwidth]{img/ROE_dimensiones_optimas.png}
    \end{center}

    \begin{itemize}
        \item<3-> $L_{teorico} = 44.92 \ mm \Rightarrow L_{optimo} = 45.16 \ mm$
        \item<4-> $y_{0,teorico} = 16 \ mm \Rightarrow y_{0,optimo} = 14,23 \ mm$
    \end{itemize}
\end{frame}

%% ========================================================
\begin{frame}{Diagrama de Radiación 3D}
\begin{itemize}
    \item<1-> Diagrama de Radiación 3D normalizado, directividad.
    \item<2-> $f = 1575.42 \ MHz$
    \item<3-> Mínimo: $-22.85 \ dBi$
\end{itemize}
    \begin{center}
        \onslide<4->\includegraphics[width=0.7\textwidth]{img/diagrama_3D.png}
    \end{center}
\end{frame}

%% ========================================================
\begin{frame}{Corte del Diagrama de Radiación 1}
\begin{columns}
    \column{0.4\textwidth}
        \includegraphics[width=\textwidth]{img/corte_XZ.png}
    \column{0.6\textwidth}
        \includegraphics[width=\textwidth]{img/corte_XZ_simulacion.png}
\end{columns}
\end{frame}

%% ========================================================
\begin{frame}{Corte del Diagrama de Radiación 2}
\begin{columns}
    \column{0.4\textwidth}
        \includegraphics[width=\textwidth]{img/corte_YZ.png}
    \column{0.6\textwidth}
        \includegraphics[width=\textwidth]{img/corte_YZ_simulacion.png}
\end{columns}
\end{frame}

%% ========================================================
\begin{frame}{Comparación con el Modelo de Cavidad}
\begin{columns}
    \column{0.5\textwidth}
        \includegraphics[width=\textwidth]{img/corte_XZ_simulacion_vs_teoria.png}
    \column{0.5\textwidth}
        \includegraphics[width=\textwidth]{img/corte_YZ_simulacion_vs_teoria.png}
\end{columns}
\begin{itemize}
    \item<2-> Se observa el efecto del plano de tierra finito.
    \item<3-> Con este diseño, se pasa a la construcción de la antena.
\end{itemize}
\end{frame}

%% ========================================================
\subsection{Construcción de la Antena}
\begin{frame}{Dimensiones para la Construcción de la Antena}
    \begin{itemize}
        \item<1-> En las simulaciones se vio que un cambio de unos milímetros afecta bastante la frecuencia de resonancia y la adaptación.
        \item<2-> Es difícil construir una antena con una longitud exacta de $L=45.16 \ mm$!
        \item<3-> Se hará una juste manual, tanto de $L$ como de $y_0$, por prueba y error, hasta lograr resultados aceptables.
    \end{itemize}
\end{frame}

%% ========================================================
\begin{frame}{Dimensiones para la Construcción de la Antena}
    \begin{itemize}
        \item<1-> Software \textit{KiCad PCB Editor} para diseño de la antena.
    \end{itemize}
\begin{columns}
    \column{0.45\textwidth}
        \begin{itemize}
            \item<3-> $L_{opt} = 45.16 \ mm$
            \item<3-> $L_{PCB} = 42 \ mm < L_{opt}$
        \end{itemize}
        \vspace{1cm}
        \begin{itemize}
            \item<4-> $y_{0,opt} = 14.23 \ mm$
            \item<4-> $y_{0,PCB} = 17 \ mm > y_{0,opt}$
        \end{itemize}
    \column{0.55\textwidth}
        \onslide<2->\includegraphics[width=\textwidth]{img/PCB_kicad.png}
\end{columns}
\end{frame}

%% ========================================================
\begin{frame}{Antena Construida}

\begin{itemize}
    \item<1-> Se construyó la antena en un PCB doble faz, FR4.
    \item<2-> Construcción con el método del planchado.
    \item<3-> Se soldó un conector SMA hembra en el borde del PCB.
\end{itemize}
\end{frame}

%% ========================================================
\begin{frame}{Antena Construida}
\begin{columns}
    \column{0.5\textwidth}
        \includegraphics[width=\textwidth]{img/foto_parche_pcb.jpeg}
    \column{0.5\textwidth}
        \includegraphics[width=\textwidth]{img/foto_parche_pcb_atras.jpeg}
\end{columns}
\end{frame}

%% ========================================================
\begin{frame}{Antena Construida}
\begin{columns}
    \column{0.5\textwidth}
        \includegraphics[width=\textwidth]{img/foto_parche_pcb_perspectiva.jpeg}
    \column{0.5\textwidth}
        \includegraphics[width=\textwidth]{img/foto_parche_pcb_conector.jpeg}
\end{columns}
\end{frame}