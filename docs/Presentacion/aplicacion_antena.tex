\subsection{GPS}
%% ========================================================
% \begin{frame}{GPS: Global Positioning System}
% \begin{columns}
%     \column{0.6\textwidth}
%         \begin{itemize}           
%             \item Sistema satelital: 32 satélites a 20200 km, órbita de 11 hs y 58 minutos, \textit{Space Segment}.
%             \item Estimar \textbf{posición} y \textbf{velocidad} de un usuario, en la superficie de la Tierra.
%             \item Exactitud de posición en el orden de las decenas de metros.
%             \item También sirve para obtener con exactitud, una \textbf{referencia temporal}.
%             \item GPS es mantenido por EE.UU. Existen otros, como GLONASS (Rusia), Galileo (UE), Beidou (China).
%         \end{itemize}

%     \column{0.4\textwidth}
%             \includegraphics[width=\textwidth]{img/constelacion_GPS.jpg}
% \end{columns}
% \end{frame}
%% ========================================================
\begin{frame}{Señales de GPS}
\begin{columns}
    \column{0.5\textwidth}
        \begin{itemize}
            \item<2-> Sistema satelital: 32 satélites a 20200 km, órbita de 11 hs y 58 minutos.
            \item<3-> Cada satélite emite señales en distintas frecuencias, en banda L (1 GHz - 2 GHz).
            \item<4-> L1: $1575.42$ MHz, L2: $1227.60$ MHz y L5: $1176.45$ MHz.
            \item<5-> En cada banda se modulan \textbf{códigos}, algunos de uso civil (abiertos) y otros de uso militar (encriptados).
        \end{itemize}

    \column{0.5\textwidth}
        \onslide<1->\includegraphics[width=\textwidth]{img/satelite_L1_L2.png}
        
\end{columns}    
\end{frame}
%% ========================================================
% \begin{frame}{Estimación de la Distancia al Satélite}

% \begin{itemize}
%     \item El código C/A (coarse/acquisition) es el típico utilizado por dispositivos comerciales (Ej.: celulares).
%     \item Se emite en L1, $1575.42$ MHz.
%     \item Secuencia de bits pseudo-aleatoria, única para cada satélite.
% \end{itemize}

% \begin{figure}
%     \centering
%     \includegraphics[width=0.7\textwidth]{img/correlacion_CA.png}
%     \caption{El receptor demodula la señal y obtiene el código. A su vez, genera una réplica local. A través de la correlación entre ambas, obtiene el \textit{delay} desde que salió la señal del satélite hasta que llega al receptor. Finalmente, utilizando la velocidad de la luz, estima la distancia al satélite.}
%     \label{fig:my_label}
% \end{figure}
% \end{frame}

%% ========================================================          
% \begin{frame}{Estimación de la Posición}

% \begin{itemize}
%     \item Repitiendo esto con 3 satélites de la constelación, se estima la posición.
%     \item Si se utiliza un cuarto satélite, se puede mejorar la estimación.
% \end{itemize}

% \begin{center}
%     \includegraphics[width=\textwidth]{img/trilateration.png}
% \end{center}
% \end{frame}

%% ======================================================== 
\subsection{Requerimientos de la Antena Receptora}

\begin{frame}{Frecuencia de Operación: 1575,42 MHz}

\begin{itemize}
    \item<1-> Frecuencia de operación: para hacer una estimación de la posición, la antena deberá recibir señales en banda L1: $\mathbf{1575.42 \ MHz}$.
    \item<2-> La banda L2 también contiene un código de uso comercial.
    \item<3-> Se elige utilizar una sola frecuencia de operación, para simplificar el diseño.
\end{itemize}
\begin{center}
    \onslide<4->\includegraphics[width=0.7\textwidth]{img/bandas_GNSS.png}
\end{center}

\end{frame}

%% ======================================================== 
\begin{frame}{Ancho de Banda}

\begin{itemize}
    \item<1-> Información útil: código C/A y código P.
    \item<2-> El código P tiene la mayor parte de su energía en una \textbf{banda de aproximadamente} $\mathbf{30}$ \textbf{MHz}.
\end{itemize}
\begin{center}
    \onslide<3->\includegraphics[width=0.5\textwidth]{img/ancho_de_banda.png}
\end{center}

\begin{center}
\onslide<4->\textbf{A priori no se consideró el ancho de banda como un aspecto de diseño de la antena.}
\end{center}
\end{frame}

%% ======================================================== 
\begin{frame}{Directividad}

\begin{itemize}
    \item<1-> Recibir señales de al menos 4 satélites al mismo tiempo.
    \item<2-> Diagrama de radiación \textbf{omni-direccional}.
\end{itemize}

\begin{center}
\onslide<3->\includegraphics[width=0.7\textwidth]{img/requerimiento_diagrama_radiacion.png}
\end{center}

\end{frame}

%% ======================================================== 
\begin{frame}{ROE}

\begin{center}
   Para la frecuencia de operación, se pide lograr una buena adaptación: 
\end{center}

\vspace{0.5cm}

\begin{center}
    ROE < 2
\end{center}

\end{frame}