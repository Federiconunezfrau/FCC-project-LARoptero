\begin{frame}{Conclusiones}
    \begin{enumerate}[<+->]
        \item [1] Se pudo construir la antena, cumpliendo con los requerimientos de GPS.
        \item [2] A partir de los modelos de cavidad resonante y de línea de transmisión, se pudo entender el mecanismo de funcionamiento de la antena parche.
        \item [3] Se plantearon dimensiones para la antena, según los requerimientos.
        \item [4] Estas dimensiones fueron modificadas parcialmente con una simulación en el software FEKO.
        \item [5] El modelo teórico modela el comportamiento de la antena.
    \end{enumerate}
\end{frame}

%% ========================================================
\begin{frame}{Conclusiones}
    \begin{enumerate}[<+->]
        \item [6] A partir de la simulación se pudieron entender los mecanismos de adaptación y de sintonización de la antena, y sus complejidades.
        \item [7] Se observaron diferencias entre la medición y la simulación de la ROE, principalmente con el ancho de banda.
        \item [8] Es probable que deba buscarse otra manera de simular la alimentación de la antena, que represente mejor la realidad.
        \item [9] Por último, las mediciones del diagrama de radiación coinciden con la simulación.
        \item [10] Se concluye que, para reducir el espacio de la antena, sería mejor utilizar un dieléctrico con un $\varepsilon_r$ más grande.
        \item [11] De esta forma el plano de tierra no debería ser tan grande, si es que se quiere disminuir el \textit{backscattering}.
    \end{enumerate}
\end{frame}