\subsection{Modelos de Cavidad Resonante}
%% ========================================================
\begin{frame}{Cavidad Resonante}

\begin{itemize}
    \item<1-> Las líneas de campo eléctrico dentro del parche son como en un microstrip:
\end{itemize}

\begin{center}
    \onslide<2->\includegraphics[width=0.5\textwidth]{img/lineas_campo_parche.png}
\end{center}

\begin{itemize}
    \item<3-> $\lambda = \frac{c}{f} = \frac{c_0}{\sqrt{\epsilon_r}} \frac{1}{f} = \frac{300 \ 10^6 \ m/s}{\sqrt{4,4}} \frac{1}{1575.42 \ MHz} = 90.8 \ mm$
    \item<4-> $h \ll \lambda \Rightarrow$ se asume que el módulo de campo eléctrico es constante en la dirección vertical.
    \item<5-> $h \ll W, L \Rightarrow$ se asume que el campo eléctrico es de dirección vertical en todo el parche.
\end{itemize}
    
\end{frame}

%% ========================================================
% \begin{frame}{Campos Dentro de la Cavidad}

% \begin{columns}
%     \column{0.5\textwidth}
%         \begin{itemize}
%             \item Solo se consideran modos $TM^x$, es decir, con $H_x = 0$.
%             \item Se plantea el potencial magnético $A_x$, junto con las condiciones de contorno: $\nabla^2 A_x + \beta^2 A_x = 0$
%             \item $H_y(x;y;z=0,W) = 0$, $H_z(x;y=L,0;z) = 0$, $E_y(x = 0,h;y;z)=0$
%         \end{itemize}
%     \column{0.5\textwidth}
%             \includegraphics[width=\textwidth]{img/cavidad.png}
% \end{columns}
% \end{frame}

\begin{frame}{Campos Dentro de la Cavidad}

\begin{columns}
    \column{0.7\textwidth}
    \begin{itemize}
        \item<2-> Potencial magnético dentro de la cavidad:
    \end{itemize}
    \begin{center}
        \onslide<3->$A_x = A_{mnp} cos(k_x \ x) cos(k_y \ y) cos(k_z \ z)$\\
    \end{center}
    \begin{center}
        \onslide<3->$k_x = \left( \frac{m \pi}{h} \right)$ \ $k_y = \left( \frac{n \pi}{L} \right)$ \ $k_z = \left( \frac{p \pi}{W} \right)$
    \end{center}

    \begin{itemize}
        \item<4-> $m, n, p \in \ \mathbb{Z} \geq 0$ y representan el modo $mnp$ resonante, dentro de la cavidad:
    \end{itemize}

    \begin{center}
        \onslide<5->$f_{mnp} = \frac{1}{2 \pi \sqrt{\mu \varepsilon}}\sqrt{\left( \frac{m \pi}{h} \right)^2 + \left( \frac{n \pi}{L} \right)^2 + \left( \frac{p \pi}{W} \right)^2 }$
    \end{center}
    \column{0.3\textwidth}
            \onslide<1->\includegraphics[width=\textwidth]{img/cavidad.png}
\end{columns}
\end{frame}

%% ========================================================
% \begin{frame}{Potencial Magnético}

% \begin{itemize}
%     \item Se llega a la expresión del potencial magnético dentro de la cavidad:
% \end{itemize}

% \begin{center}
%     $A_x = A_{mnp} cos(k_x \ x) cos(k_y \ y) cos(k_z \ z)$\\
% \end{center}

% \begin{center}
%     $k_x = \left( \frac{m \pi}{h} \right)$ \ $k_y = \left( \frac{n \pi}{L} \right)$ \ $k_z = \left( \frac{p \pi}{W} \right)$
% \end{center}

% \begin{itemize}
%     \item $m, n, p \in \ \mathbb{Z} \geq 0$ y representan el modo $mnp$ resonante, dentro de la cavidad:
% \end{itemize}

% \begin{center}
%     $f_{mnp} = \frac{1}{2 \pi \sqrt{\mu \varepsilon}}\sqrt{\left( \frac{m \pi}{h} \right)^2 + \left( \frac{n \pi}{L} \right)^2 + \left( \frac{p \pi}{W} \right)^2 }$
% \end{center}

% \end{frame}

%% ========================================================
\begin{frame}{Resonancia}

\begin{itemize}
    %\item Hay muchas frecuencias a la que resuena la cavidad.
    \item<1-> Se selecciona la correspondiente al modo $TM^x_{010}$: $f_{010} = \frac{1}{2 \sqrt{\mu \epsilon}} \frac{1}{L}$
\end{itemize}

\begin{center}
    \onslide<2->\includegraphics[width=0.4\textwidth]{img/cavidad_resonante_010.png}
\end{center}

\begin{itemize}
    \item<3-> Se hallaron los campos dentro del parche $\Rightarrow$ Se puede estudiar el mecanismo de radiación.
\end{itemize}

\end{frame}

%% ========================================================
\begin{frame}{Mecanismo de Radiación}

\begin{itemize}
    \item<1-> Los 4 bordes de la cavidad se modelan como antenas de apertura (\textit{radiating slots}).
    \item<2-> Principio de Huygens: se plantean las fuentes equivalentes para las 4 aperturas de la cavidad:
\end{itemize}

\begin{center}
    \onslide<3->$\overrightarrow{J_s} = \hat{n} \times \overrightarrow{H}$ \hspace{1cm} $\overrightarrow{M_s} = -\hat{n} \times \overrightarrow{E}$
\end{center}

\begin{itemize}
    \item<4-> En las 4 aperturas, $\overrightarrow{H} = H \ \hat{n}$, $\Rightarrow$ $\overrightarrow{J_s} = \overrightarrow{0}$.
    \item<5-> Solo hay corrientes magnéticas equivalentes:
\end{itemize}

\begin{columns}
    \column{0.5\textwidth}
        \onslide<6->\includegraphics[width=0.9\textwidth]{img/corrientes_magneticas_1.png}
    \column{0.5\textwidth}
        \onslide<6->\includegraphics[width=0.9\textwidth]{img/corrientes_magneticas_2.png}
\end{columns}
\end{frame}

%% ========================================================
\begin{frame}{Aberturas Radiantes}
    \begin{itemize}
        \item<1-> Las corrientes de las aberturas radiantes están en fase y tienen la misma amplitud.
        \item<2-> Conjunto de 2 antenas \textit{broadside}, separadas una distancia $L$.
        %\item El diagrama de radiación de cada apertura es como el del dipolo.
    \end{itemize}

    \begin{columns}
        \column{0.5\textwidth}
            \begin{center}
                \onslide<3->\includegraphics[width=\textwidth]{img/corrientes_magneticas_1.png}
            \end{center}
        \column{0.5\textwidth}
            \begin{center}
                \onslide<4->\includegraphics[width=\textwidth]{img/diagrama_radiacion_cavidades.png}
            \end{center}
    \end{columns}
\end{frame}

%% ========================================================
\begin{frame}{Aberturas no Radiantes}

    \begin{itemize}
        \item<1-> Las corrientes de las aberturas están en contrafase.
        \item<3-> La radiación de estos slots es mucho menor a la de los anteriores.
        \item<4-> Por eso se los llama \textit{Non-radiating slots}.
        %\item El diagrama de radiación queda definido por las aberturas radiantes.
    \end{itemize}

\begin{center}
    \onslide<2->\includegraphics[width=0.6\textwidth]{img/corrientes_magneticas_2.png}
\end{center}
\end{frame}

%% ========================================================
\subsection{Modelo de Línea de Transmisión}
\begin{frame}{Modelo de Línea de Transmisión}
    \begin{itemize}
        \item<1-> Se modela el parche como dos aberturas, separadas por una línea de transmisión.
    \end{itemize}
    \begin{columns}
        \column{0.5\textwidth}
            \onslide<2->\includegraphics[width=\textwidth]{img/parche_modelo_transmision.png}
        \column{0.5\textwidth}
            \onslide<3->\includegraphics[width=\textwidth]{img/equivalente_linea_transmision.png}
    \end{columns}
    \begin{itemize}
        \item<3-> Las aberturas radiantes se modelan como impedancias $Z_1$ y $Z_2$.
    \end{itemize}
\end{frame}

%% ========================================================
\begin{frame}{Impedancia de Entrada}
    \begin{itemize}
        %\item En resonancia, la impedancia de entrada es puramente real.
        \item<1-> Adaptación de la impedancia de entrada del parche con el microstrip:
    \end{itemize}

   \begin{center}
       \onslide<2->\includegraphics[width=0.5\textwidth]{img/parche_inset_feed.png}
   \end{center} 

    \begin{itemize}
        \item<3-> Impedancia del microstrip: $50 \ \Omega$.
    \end{itemize}
    
\end{frame}

%% ========================================================
\begin{frame}{Adaptación}
    \begin{itemize}
        \item Modelo equivalente:
    \end{itemize}

    \begin{center}
        \includegraphics[width=0.5\textwidth]{img/equivalente_linea_transmision_adaptacion.png}
    \end{center}

    \begin{itemize}
        \item $Z_{in} = Z_{in}(y_0 = 0) \ cos^2\left( \beta \ y_0 \right)$
        \item Se modifica $y_0$ hasta lograr una impedancia de $50 \ \Omega$.
    \end{itemize}

\end{frame}